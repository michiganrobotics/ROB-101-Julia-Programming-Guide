
\begin{tcolorbox}[title=\textbf{\Large C++ vs Julia: What are the differences?}]


What is C++?\\ 
-has imperative, object-oriented and generic programming features.\\
-allows low level memory manipulation.\\
-compiles directly to a machine's native code: allowing it to be one of the fastest languages in the world.
\\\\
What is Julia?\\ 
-a high-level, high-performance dynamic programming language for technical computing.\\
-has syntax that is familiar to users of other technical computing environments. \\
-has an extensive mathematical function library.\\

\textbf{This document shows some key comparisons between C++ and Julia.}


\begin{center}
\url{https://docs.google.com/spreadsheets/d/1bU8V8lPG2kHo1k7M_ylykijzLvfGvwlCEPaZHqTTTk8/edit?usp=sharing
}
\end{center}


\end{tcolorbox}
\vspace*{.2cm}

Julia is a very high level programming language, unlike C++. Julia provides a lot of mathematical functions that C++ does not have. Hence, learning Julia with a C++ foundation is very helpful. There are, however, some critical differences between their syntax that can make it hard to go from one language to the other when you are first learning them. In the above spreadsheet, we compared how to accomplish a few key tasks in C++ and Julia. Please share with us in Piazza other commands in C++ that you wish to execute in Julia; we'll try to add them to our spreadsheet.
